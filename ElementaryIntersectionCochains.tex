\documentclass{amsart}          % Nicer than default article style:  less
                                % flashy headings, etc.

%\usepackage{amsmath,amsthm}     % Handy math stuff, theorem environments.
\usepackage{amssymb, amscd}            % Fancy math symbols.
\usepackage{pstricks,pstricks-add}	% Pretty pictures
\usepackage{float}	% Allows finer control over placement of figures than the default automatic placement style

\newcommand{\Z}{{\mathbb{Z}}}

\textwidth = 6.5 in
\textheight = 8.5 in
\oddsidemargin = 0.0 in
\evensidemargin = 0.0 in
%\topmargin = 0.0 in
%\headheight = 0.0 in
%\headsep = 0.0 in
\parskip = 0.2in
\parindent = 0.0in

\newtheorem{theorem}{Theorem}
\newtheorem{corollary}[theorem]{Corollary}
\newtheorem{proposition}[theorem]{Proposition}
\newtheorem{lemma}[theorem]{Lemma}
\newtheorem{definition}[theorem]{Definition}
\newtheorem{remark}[theorem]{Remark}
\newtheorem{example}{Example}

\newcommand{\R}{\mathbb R}
\newcommand{\Q}{\mathbb Q}
\newcommand{\colim}{{\rm colim}}
\newcommand{\C}{\mathbb C}
\newcommand{\D}{\mathcal D}
\newcommand{\E}{\mathcal E}
%\newcommand{\ker}{{\rm ker}\;}
\newcommand{\coker}{{\rm coker}\;}
\newcommand{\ext}{{\rm Ext}}
\newcommand{\injext}{{\rm InjExt}}
\newcommand{\Hom}{{\rm Hom}}

\begin{document}

\title{Elementary Intersection Theory}
%\author{Dev Sinha}
\maketitle

\section{Introduction}

We present a treatment of elementary intersection theory in algebraic topology.  
That is, we aim to show how one can  
define cochains through counting intersection with or more generally preimages of
submanifolds.  Moreover, the ``fundamental theorem''  is that the cup product of the associated
 cohomology classes is represented by the (transversal) intersection of submanifolds.  
 Students are often told that cup product is named so because of this 
 fact, but we feel it is given relatively short shrift.  It is usually proved after development of duality or a Thom isomorphism theorem.  
 We would like to put the representation of cohomology by submanifolds at the front and center of our treatment and from that deduce, 
 or at least interpret, these related isomorphisms.

Unlike chains, singular cochains tend to be more transcendental objects.  From a pedagogical  viewpoint, one can often express cycles as in terms of explicit chains, which is basically never the case for singular cocyles.  Our work remedies this discrepancy, and  allows for geometric cochain-level understanding of a number of topics in algebraic topology including duality, Thom isomorphisms, cohomology of mapping spaces
etc. 


\section{Basic definitions, and a simplified version of the fundamental theorem}

Before developing the background in differential topology needed, we illustrate the technique by using simplified definitions.

\begin{definition}
Manifold, submanifold.
\end{definition}

\begin{definition}
Tangent bundle and tangent vector of a submanifold.
\end{definition}

\begin{definition}
Transversal intersection of submanifolds.
\end{definition}
	\begin{figure}[H]
	\begin{center}
		\begin{pspicture}(0.5,-0.5)(3.5,3.5)			
		\pspolygon(-2,1)(0,1)(1,2)(-1,2)
		\psline(-.5,0)(-.5,1)
		\psline[linestyle=dashed,dash=1.3pt](-.5,1)(-.5,1.5)
		\pscircle*(-.5,1.5){1pt}
		\psline[linecolor=white,linewidth=1.2pt](-.47,2)(-.53,2)
		\psline(-.5,1.5)(-.5,3)	
		\rput(-3,1.5){\textup{a)}}
		
		
		\pspolygon(4,1)(4,-.25)(5.2,.75)(5.2,2)
		\psline[linecolor=white,linewidth=1.2pt](5.2,1.2)(5.2,2)
		\psline[linestyle=dashed,dash=1.3pt](5.2,1.2)(5.2,2)
		\pspolygon(3,1)(5,1)(6.2,2)(4.2,2)
		\psline[linecolor=white,linewidth=1.2pt](4.2,2)(5.2,2)
		\psline[linecolor=white,linewidth=1.2pt](4,1.83)(4.2,2)
		\psline[linestyle=dashed,dash=1.3pt](4,1.83)(4.2,2)
		\psline[linestyle=dashed,dash=1.3pt](4.2,2)(5.2,2)
		\pspolygon(4,1)(4,2.25)(5.2,3.25)(5.2,2)
		\rput(2,1.5){\textup{b)}}
		\end{pspicture}
	\end{center}
	\caption{Examples of transversally intersecting submanifolds of $\R^3$: \textup{a)} the plane $z=0$ and the line $x=y=0$ and \textup{b)} the planes $z=0$ and $y=0$.}
\end{figure}

\begin{definition}
Codimension.
\end{definition}

\begin{theorem}
If submanifolds intersect transversally, then the intersection is a submanifold.  The codimension is the sum of their codimensions.
\end{theorem}

We can now state a simplified version of the fundamental theorem of intersection theory.

\begin{theorem}
Let $M$ be a manifold and $W$ be a submanifold of codimension $d$ without boundary.  
\begin{enumerate}
\item There is a cohomology class $[\tau_{W}] \in H^{d}(M; \Z/2)$ represented by a cochain whose value on an embedded chain $\Delta^{d} \subset M$ which intersects $W$ transversally is the mod-two count of $\Delta^{d} \cap W$.
\item If $S$ is a submanifold whose boundary is $V \bigsqcup W$ then $[\tau_{V}] = [\tau_{W}]$.
\item If $V$ and $W$ intersect transversally then $[\tau_{V}] \cup [\tau_{W}] = [\tau_{V \cap W}]$.
\end{enumerate}
\end{theorem}

\subsection{Examples using the simplified version}
\begin{example}
$\R^{2} - 0$
\end{example}
\begin{example}
$[\tau_M] = 1$.  $[\tau_{{\rm point}}]$.
\end{example}
\begin{example}
$\R^{3}$ with the z-axis removed along with either a linked or unlinked $S^{1}$.
\end{example}
\begin{example}
A two-holed torus
\end{example}
\begin{example}
$\R P^{n}$ and application to no retraction.
\end{example}

\subsection{Comparison with other approaches to cochains}

Cochains are transcendental data (a value on every chain).  But some representations are more finite.

Focus on torus example.  Both simplicial and de Rham  generally assess ``tolls'' for chains.  Both can be made ``concentrated'' as one
crosses a submanifold.  


\section{Transversality and intersection cochains}

\begin{definition}
Transversality more generally.
\end{definition}

\begin{definition}
Let $M$ be a manifold and $f : W \to M$ an immersion. Then $C^{\pitchfork W}_{*}(M)$ is the subset of $C_{*}(M)$ generated by the chains whose image is transversal to $f$. More generally, if $S$ is a  finite set of immersed submanifolds of $M$ then $C^{\pitchfork S}_{*}$ is generated by the chains whose image is transversal to each $W \in S$.
\end{definition}

\begin{theorem}\label{T:qi}
For any finite collection $S$ of submanifolds of $M$, the inclusion $C^{\pitchfork S}_{*}(M) \hookrightarrow C_{*}(M)$ admits a chain deformation retraction.
\end{theorem}


\begin{proof} We wish to construct a map $j:C_{*}(M) \rightarrow C^{\pitchfork S}_{*}(M)$ such that $j$ is the identity on $C^{\pitchfork S}_{*}(M)$ and 
$i \circ j$ is chain homotopic to the identity on $C_{*}(M)$. Given $\sigma:\Delta^d \rightarrow M$, we can cover its image with a finite number of 
open sets $\{U_i\}$ in $M$ diffeomorphic to open subsets of $\R^k$, and we can choose these so that on the overlap the transition from one 
diffeomorphism to another is smooth. We proceed inductively: \\If $e^0_k \subset U_k$ is not transversal to $S$, then $e^0_k \in S$ and by 
Sard's Theorem we may choose a path $\gamma_k$ in $U_k$ from $e^0_k$ to a point not in $S$. If $e^0_k \pitchfork S$, then $\gamma_k$ is the 
constant map. Using the consistent cover of $\sigma$, we may extend this to a map $\bar{\sigma}:\Delta^d \rightarrow M$ such that $\bar{\sigma}$ 
on the 0-cells is defined by the endpoints of the $\gamma_k$'s. This gives a map 
$\left(\Delta^d \times \{0,1\} \right)\cup \left(\{e^0_k\} \times I \right)\rightarrow M$, and simplicial complexes have the homotopy extension property, 
so can extend this to a homotopy $F: \Delta^d \times I \rightarrow M$. Then by Theorem~\ref{T:qr}, there exists a smooth homotopy $F'$ which 
agrees with $F$ on the 0-cells and $\sigma'=\left.F'\right|_{\Delta^d \times \{1\}}$ is transversal to $S$.\\
Now assume that we have a homotopy $\partial$G on the cells of dimension $<q$. If $e^q_k $ is not transversal to $S$, then we........  
[[Use relative transversality
extension.]]


We define $j(\sigma) = \sigma'$ and since $\left.j\right|_{C^{\pitchfork S}_{*}(M)}=id$ and the two chains are always homotopic, $C^{\pitchfork S}_{*}(M) \hookrightarrow C_{*}(M)$ is a chain deformation retraction and their homology groups are isomorphic.
\end{proof}

We now define our main objects of study.
\begin{definition}
If $W \in S$ then $\tau_{W} \in C^{*}_{\pitchfork S}(M)$ is the cochain defined by $\tau_W (\sigma) = \# P,$  where $P$ is
the  zero-manifold defined as the following pullback
$$
\begin{CD}
P @>>> W\\
@VVV  @VfVV\\
\Delta^n @>\sigma>> M.
\end{CD}
$$
The cardinality $\# P$ is to be taken as mod-two unless $M$ and $W$ are oriented, in which case $\Delta^n$ is given the  orientation with say the differences between vertices labelled by $1, \cdots, n$ and vertex $0$ (or the barycenter) serving as basis vectors, and $P$ is then oriented accordingly.
\end{definition}

We call these Thom classes, because in the standard development they are (the pushforwards of)  Thom classes of the normal bundles.  But we will use 
them to understand both pushforward and Thom classes, rather than the other way around.

\begin{theorem}
If $\partial V = W$ then $\delta \tau_{V} = \tau_{W} \in C^{*}_{\pitchfork \{V, W\}}$.  In particular, if $V$ has no boundary then $\tau_{V}$ is a cocycle.
\end{theorem}

\begin{proof}[Sketch, for now]
This comes down to the classification of one-manifolds.  Consider $\sigma : \Delta^{n+1} \to M$.  
Look at pull back of $W$ through $\sigma$ to get a one-manifold with boundary.  Some boundary points are on $\partial \Delta^{n+1}$ - 
counting those gives 
 $\delta \tau_V (\sigma)  = \tau_V (\partial \sigma)$.  Other boundary points are in the interior, which come from the preimage of 
$\partial V = W$, and thus correspond to $\tau_{W} (\sigma)$.
\end{proof}

\begin{figure}[H]
	\begin{center}
		\begin{pspicture}(0.5,-0.5)(3.5,3.5)			
		\pspolygon[linewidth=1.2pt](0,0)(4,0)(2,3.46410161514)
		\psarc[linecolor=orange!90,linewidth=1.2pt](0.66666,1.15470053838){1.5em}{-120}{60}
		\pscurve[linecolor=orange!90,linewidth=1.2pt](1.33333,2.3094017676)(1.7,2)(1.9,2.1)(2.45,1.7)
		\psdot[linecolor=red!90,fillcolor=red!90](0.41,0.7)
		\psdot[linecolor=red!90,fillcolor=red!90](0.93,1.6)
		\psdot[linecolor=red!90,fillcolor=red!90](1.3333,2.3094)
		\psdot[linecolor=blue!70!violet!60,fillcolor=red!50](2.44,1.7)
		\psline[linecolor=orange!90,linewidth=1.2pt](2,1.5)(3,1)
		\psdot[linecolor=blue!80!violet!60,fillcolor=red!50](2,1.5)
		\psdot[linecolor=blue!80!violet!60,fillcolor=red!50](3,1)
		\psccurve[linecolor=orange!90,fillcolor=red!90](1.5,0.5)(1.8,0.35)(2,0.5)(2.2,0.35)(2.5,0.5)(2,1)
		\end{pspicture}
	\end{center}
\caption{Example in the case $n=1$ with $\sigma^{-1}(V)$ in orange, points in $\left(\left.\sigma\right|_{\partial\Delta^2}\right)^{-1}(V)$ marked in red, and those in $\sigma^{-1}(\partial V)$ marked in blue.}
\end{figure}

By abuse, we denote by $[\tau_{W}]$ the corresponding singular cohomology class under the isomorphism established in Theorem~\ref{T:qi}.

\begin{corollary}
If $\partial V = W \sqcup W'$ then $[\tau_W] = [\tau_{W'}]$.
\end{corollary}

This theorem and its
corollary recovers a classical view of cohomology, as represented by submanifolds with the relation defined by submanifolds with boundary.  But in fact
not all cocycles are represented in this way, which was a famous question of Steenrod addressed by Renee Thom.  Our theory allows for linear combinations
of submanifolds, and even manifolds with corners below, in which case all cohomology is representable.  The submanifold point of view led Thom
to initiate cobordism theory, which gives generalized cohomology theories.

Induced maps are a basic ingredient of cohomology theory.  In the case of Thom classes, they are geometrically defined on the cochain level.  The
proof of the following is immediate.

\begin{proposition}\label{pullback}
If $f : M \to N$ is transverse to $W \subset N$ then $f^{\#}(\tau_{W}) = \tau_{f^{-1} W}$.
\end{proposition}



\section{Intersection and cup product}

In this section we prove the following.

\begin{theorem}
If $V$ and $W$ intersect transversally, and the intersection is given the orientation... then $[\tau_{V}] \smile [\tau_W] = [\tau_{V \cap W}]$.
\end{theorem}

Recall that the cup product is essentially induced by the diagonal map $X \overset{\Delta}{\to} X \times X$.  Under the standard
formulaic approach, the cohomology of a product $X \times Y$ is developed through the external cup product, namely $C^*(X) \otimes C^*(Y)$
maps to $C^*(X \times Y)$ by sending $\alpha  \otimes \beta$ to $p^* \alpha \cup q^*\beta$, where $p$ is the projection from 
$X \times Y$ to $X$ and $q$ is the projection to $Y$.

\begin{proposition}\label{externalcup}
The external cup product sends $[\tau_V] \otimes [\tau_W]$ to $[\tau_{V \times W}]$
\end{proposition}

\begin{proof}[Proof of the Fundamental Theorem, based on Proposition~\ref{externalcup}]
If $V \pitchfork W$ then $\Delta \pitchfork V \times W \subset X \times X$.  Apply Proposition~\ref{pullback}.
\end{proof}

\begin{proof}[Proof of Proposition~\ref{externalcup}]
Recall K\"unneth theorem that  $C_*(X) \otimes C_*(Y) \to C_*(X \times Y)$ given by (subdividing) products of simplices in $X$ and $Y$ is an
isomorphism on homology.  Then show that $p^* \tau_V \cup q^* \tau_W$ and $\tau_{V \times W}$ agree on this subcomplex, which is has an immediate
geometric proof.
\end{proof}

\section{Examples}

We make some calculations 
It is more standard to establish fundamental classes in homology, rather than Thom classes.  But we see that Thom classes are just as natural if not moreso.
The following is immediate from the definitions.

\begin{proposition}
The value of a Thom class on  a fundamental class given by triangulating a submanifold is given by intersection.
\end{proposition}


\subsection{Projective spaces}

\subsection{Grassmannians}

\subsection{Complements of submanifolds which are boundaries}

\subsection{Configuration spaces}

\section{Applications: wrong-way maps, duality, suspensions and Thom isomorphism}

\section{Looking forward: characteristic classes, loop spaces, and cobordism.}

\end{document}

 
