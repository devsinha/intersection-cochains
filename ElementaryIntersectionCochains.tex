\documentclass{amsart}          % Nicer than default article style:  less
                                % flashy headings, etc.

%\usepackage{amsmath,amsthm}     % Handy math stuff, theorem environments.
\usepackage{amssymb, amscd}            % Fancy math symbols.
\usepackage{pstricks,pstricks-add}	% Make pretty pictures with PostScript
\usepackage{float}	% Allows finer control over placement of figures than the default automatic placement style
\usepackage{imakeidx} % Index
\usepackage{todonotes} % Allows one to add both in-line and marginal notes as well as a list of 'to-dos' Example syntax: \todo[color=blue!25,inline]{Edit.}
%this example produces an in-line to-do which says "Edit." and which is colored blue with 25% color saturation. Omitting 'inline' from the syntax produces a to-do in the margin. Warning: the default color is an awful shade of violently bright orange!

\newcommand{\Z}{{\mathbb{Z}}}

\textwidth = 6.5 in
\textheight = 8.5 in
\oddsidemargin = 0.0 in
\evensidemargin = 0.0 in
%\topmargin = 0.0 in
%\headheight = 0.0 in
%\headsep = 0.0 in
\parskip = 0.2in
\parindent = 0.0in

\newtheorem{theorem}{Theorem}
\newtheorem{corollary}[theorem]{Corollary}
\newtheorem{proposition}[theorem]{Proposition}
\newtheorem{lemma}[theorem]{Lemma}
\newtheorem{definition}[theorem]{Definition}
\newtheorem{remark}[theorem]{Remark}
\newtheorem{example}[theorem]{Example}
\newtheorem{exercise}[theorem]{Exercise}


\newcommand{\R}{\mathbb R}
\newcommand{\I}{\mathbb I}
\newcommand{\Q}{\mathbb Q}
\newcommand{\colim}{{\rm colim}}
\newcommand{\C}{\mathbb C}
\newcommand{\D}{\mathcal D}
\newcommand{\E}{\mathcal E}
%\newcommand{\ker}{{\rm ker}\;}
\newcommand{\coker}{{\rm coker}\;}
\newcommand{\ext}{{\rm Ext}}
\newcommand{\injext}{{\rm InjExt}}
\newcommand{\Hom}{{\rm Hom}}
\newcommand{\RP}{\mathbb{R}\mathrm{P}}
\newcommand{\CP}{\mathbb{C}\mathrm{P}}
\newcommand{\codim}{{\rm codim}}
\newcommand{\Gr}{\mathrm{Gr}}
\newcommand{\pdiff}[2]{\frac{\partial #1}{\partial #2}}
\newcommand{\diff}[2]{\frac{d #1}{d #2}}

\makeindex
\begin{document}

\title{Elementary Intersection Theory}
%\author{Dev Sinha}
\maketitle

\section{Introduction}

We present a treatment of elementary intersection theory in algebraic topology.  
That is, we aim to show how one can  
define cochains through counting intersection with submanifolds.  
Moreover, the ``fundamental theorem''  is that the cup product of the associated
 cohomology classes is represented by the (transversal) intersection of submanifolds.  
 Students are often told that cup product is named so because of this 
 fact, but we feel it is given relatively short shrift.  It is usually proved after development of duality or a Thom isomorphism theorem.  
 We would like to put the representation of cohomology by submanifolds at the front and center of our treatment and from that deduce, 
 or at least interpret, these related isomorphisms.

We can view our aims both pedagogical and mathematical perspectives.
Unlike chains, singular cochains tend to be more transcendental objects.  
From a pedagogical  viewpoint, one can often express cycles as in terms of explicit chains, which is basically never the case 
for singular cocyles.  Our work remedies this discrepancy, and  allows for geometric cochain-level understanding of a 
number of topics in algebraic topology including duality, Thom isomorphisms, cohomology of mapping spaces etc.   
Mathematically, we aim for the following.

\begin{definition}
A complete presentation of a free module $V$ is a choice of spanning sets for $V$ and a collection of ${\rm Hom}(V, M_\alpha)$
which distinguish elements of $V$, along with a calculation of the pairings between these spanning sets.
\end{definition}

\begin{exercise}
Find a minimal collection of $M_\alpha$ when $V$ is the $n$th homology of  a space, which is finitely generated.
\end{exercise}

Our mathematical goal is to find complete presentations for homology, going beyond the calculation of isomorphism class.  
Such can be helpful in for example calculating homomorphisms between homology groups induced by maps of spaces,
as well as in calculating cohomology rings, a primary application.  

Complete presentations are possible through $\Delta$-complex
presentations of spaces, but such can get too large to manage (by hand or even computer), and we find them  to be less compelling
than finding (sub)manifolds when the latter are available.  
Any finite CW-complex is homotopy equivalent to a manifold with boundary, by  embedding the complex in Euclidean space
and taking the closure of a nice neighborhood which retracts onto the complex, so while seemingly specialized these techniques
can shed light on homotopy theory for limits and colimits of finite CW-complexes and in particular all CW complexes themselves.\newpage

\tableofcontents
\newpage

\section{Basic definitions, and a simplified version of the fundamental theorem}

Before developing the background in differential topology needed, we illustrate the technique by using simplified definitions.

\begin{definition}
	A subset $M$ of Euclidean space $\R^n$ is called a (smooth) $k$-dimensional manifold if it is locally diffeomorphic to $\R^k$; that is, for every point in $M$ there is a neighborhood of that point which is diffeomorphic to $\R^k$.  
\end{definition}

We often omit the word ``smooth," and that qualifier will be assumed throughout these notes unless otherwise stated. There are a number of different ways to describe manifolds, and in particular we will find it useful to describe them via ``parametrizations" and via solution sets to various equations. The latter definition takes more machinery to justify, but we will illustrate an early example for motivation.

Establishing local diffeomorphisms which make a set $M\subset \R^n$ into a manifold does more than simply establishing that $M$ is a manifold, but provides machinery allowing us to establish other properties of $M$. We call a map $\phi: \R^k \rightarrow M$ which is a diffeomorphism onto its image $U$ a \textit{parametrization} of its image. We could equivalently replace $\R^k$ with an open subset of $\R^k$ if we desired. The inverse map $\phi^{-1}:U\rightarrow \R^k$ is called a \textit{coordinate system} on $U$, and the pairing $(U, \phi)$ is called a \textit{chart}. The existence of a collection of charts $\{ (U_\alpha, \varphi_\alpha) | \alpha \in A\}$ such that $\bigcup_{\alpha\in A}U_\alpha = M$ is equivalent to the condition that $M$ is a manifold. Such a collection is called an \textit{atlas}. In the case where $M$ is compact, every atlas can be reduced to finitely many charts. We can use parametrizations and charts to prove that a space is a manifold by covering it with a collection of diffeomorphisms from $\R^k$.

\begin{example}
	$S^1 = \{(x,y)\in \R^2 | x^2 + y^2=1\}$ is a manifold of dimension $1$. \\
	
	We can exhibit four parametrizing maps from $(-1,1)$ into $S^1$. Given $x \in (-1,1)$, the map onto the upper open semicircle given by $x\mapsto(x, \sqrt{1-x^2})$ is smooth. We can define a similar map on to the lower open semicircle given by $x \mapsto (x, 1-\sqrt{1-x^2})$. These two maps alone \textit{do not} show that $S^1$ is a manifold: The points at $(1,0)$ and $(-1,0)$ are not in the image of either map. Two similar maps sending $(-1,1)$ smoothly onto the left and right open semicircles completes a parametrization of $S^1$.  
	
	[image here]
	
	Since $(-1,1)$ is diffeomorphic to $\R$ this proves that $S^1$ is a manifold and establishes that the dimension of $S^1$ is 1. \\
\end{example} 

\begin{example}
	$\RP^n$ can be covered by precisely $n+1$ charts. Consider $\RP^n$ as $\R^{n+1} \backslash \{0\}$ under the equivalence relation $(x_1,..., x_{n+1})\sim (ax_1,...,ax_{n+1})$ for $a \in \R$. Define $U_i$ as the subset of $\RP^n$ given by all elements whose $i$th coordinate is nonzero. In $U_i$, each element has a unique representative such that $x_i=1$, written as $[x_0,...,x_{i-1},1,x_{i+1},...x_{n+1}]$. This gives us a way to embed the charts $U_i$ as copies of $\R^n$ into $\R^{n+1}$ by setting the $i$th component equal to 1. This establishes that $\RP^n$ is an $n$-manifold. A visualization of the subspaces of $\RP^2$  which result from this process, depicted on the well-known cellular structure, is shown below.
	
	[image here]  
	
\end{example}

\begin{example}
	The set $S$ of all points in $\R^2$ given by ${(x,y)|xy=\epsilon, \epsilon\neq 0}$ is a $1$-manifold parametrized by the two maps $x \mapsto \frac{\epsilon}{x}$ for $x\in (0,\infty)$ and $x \in (-\infty,0)$. The solution set when $\epsilon = 0$ is not a manifold, however, because no neighborhood containing $(0,0)$ is locally diffeomorphic to $\R$: Removing the point $(0,0)$ from any such neighborhood results in four connected components, not two.  
	
	[image here]
\end{example}

\begin{definition}
	A subset $N\subseteq M\subseteq \R^n$ is called a submanifold of $M$ if $N$ is also a manifold.
\end{definition}

Note, in particular, that if $N$ is a submanifold of $M$ that $N$ and $M$ do not need to have the same dimension, and it is generally more interesting when they do not: it is a fact that if $N$ is a compact submanifold of a connected manifold $M$ with the same dimension, then $N=M$. 

We next present an initial discussion of tangent spaces, for manifolds embedded in Euclidean space. Tangent spaces are fundamental to many of the definitions related to intersection theory that we will see later.

Let $M$ be a smooth submanifold of $\R^n$. 
\begin{definition}
Let $x\in M$ be a point. Fix a local parametrization $\phi:U\to X$ mapping an open subset $U\subseteq\R^k$ to a neighbourhood of $x$. We take $\phi(0)=x$ without loss of generality. A \emph{tangent vector at $x$} is an element of the vector space $T_x M$, which is defined to be the image of the map $d\phi_0:U\to X$.
\end{definition}
We note that this definition of $T_xM$ makes sense, since $U$ and $X$ are subsets of the vector spaces $\R^k$ and $\R^n$ respectively, and $d\phi$ is a linear map.
\begin{definition}
Our first definition of \emph{the tangent bundle of $M$} is given as the following subset of $M\times\R^n$:
\[
TM:=
\{
(x,v)~|~x\in M,~v\in T_x M
\}.
\]
\end{definition}
\begin{definition}
	If a manifold $M$ is presented abstractly, rather than as a subset of $\R^n$, then a more natural --- though far less intuitive --- definition of the tangent space $T_xM$ is that it is the \textit{space of derivations at $x$}, which is to say the space of all linear maps $v:C^\infty(M)\to\R$ which satisfy\begin{align*}
	v(fg)=f(x)v(g)+g(x)v(f)
	\end{align*}for all $f,g\in C^\infty(M)$.
\end{definition}
\begin{example}
	Suppose that $v\in\R^n$ and define $D_v:C^\infty(\R^n)\to C^\infty(\R^n)$ by $(D_vf)(x)=\left.\diff{}{t}f(x+tv)\right|_{t=0}$, then, given $x\in\R^n$, the map $\left.D_v\vphantom{\diff{}{x}}\right|_x:f\mapsto (D_vf)(x)$ is a derivation at $x$.
\end{example}
\begin{exercise}
	Verify that $\left.D_v\vphantom{\diff{}{x}}\right|_x$ is a derivation at $x$ for all $x,v\in\R^n$.
\end{exercise}
\begin{exercise}
	Show that any linear combination of derivations at $x$ is again a derivation at $x$.
\end{exercise}
\begin{proposition}
	If $M$ is a smooth $n$-manifold and $(U,\varphi)$ is a co\"{o}rdinate chart on $M$, then, for any $x\in U$, the derivations\begin{align*}
	\left.\pdiff{}{x^i}\right|_x=\left(d\varphi_x\right)^{-1}\left(\left.\pdiff{}{x^i}\right|_{\varphi(x)}\right)
	\end{align*}form a basis for $T_xM$, where $\left.\pdiff{}{x^i}\right|_{\varphi(x)}=\left.D_{e_i}\right|_{\varphi(x)}$ and $e_i=(0,\dots,1,\dots,0)$ is the $i^\text{th}$ standard unit basis vector for $\R^n$. Moreover, given $f\in C^\infty(U)$,\begin{align*}
	\left.\pdiff{}{x^i}\right|_xf=\pdiff{\widehat{f}}{x^i}(\varphi(x)),
	\end{align*}where $\widehat{f}=f\circ\varphi^{-1}$ is the co\"{o}rdinate representation of $f$ in $(U,\varphi)$.
\end{proposition}
\begin{proof}
	Exercise.
\end{proof}
\begin{proposition}
	If $M$ is a smooth $n$-manifold, the tangent bundle $TM$ admits the structure of a smooth $2n$-manifold with respect to which the canonical projection map $\pi:TM\to M$ is smooth.
\end{proposition}
\begin{proof}[Sketch]
	Choose an atlas of charts $\{(U_\alpha,\varphi_\alpha)\}$ for $M$. Fixing a chart $(U_\alpha,\varphi_\alpha)$, say with co\"{o}rdinate functions $x^1,\dots,x^n$, we have that $\pi^{-1}(U_\alpha)=\{(x,v):x\in U_\alpha,v\in T_xM\}$. Define a map $\widetilde{\varphi}_\alpha:\pi^{-1}(U_\alpha)\to\R^{2n}$ by\begin{align*}
	\widetilde{\varphi}_\alpha\left(\sum_{i}v^i\left.\pdiff{}{x^i}\right|_p\right)=\left(x^1(p),\dots,x^n(p),v^1,\dots,v^n\right).
	\end{align*}Note that $\widetilde{\varphi}_\alpha$ is a bijection onto its image with inverse\begin{align*}
	\widetilde{\varphi}^{-1}_\alpha\left(x^1,\dots,x^n,v^1,\dots,v^n\right)=\sum_{i}v^i\left.\pdiff{}{x^i}\right|_{\varphi^{-1}(x^1,\dots,x^n)}
	\end{align*} and $\widetilde{\varphi}_\alpha(\pi^{-1}(U_\alpha))=\varphi_\alpha(U_\alpha)\times\R^n$ which is open since $(U_\alpha,\varphi_\alpha)$ is a chart. One may then show that if we are given two charts $(U_\alpha,\varphi_\alpha)$ and $(U_\beta,\varphi_\beta)$ with co\"{o}rdinate functions $\{x^i\}$ and $\{y^j\}$, respectively, then the transition function $\widetilde{\varphi}_{\alpha\beta}=\widetilde{\varphi}_\beta\circ\widetilde{\varphi}_\alpha:\varphi_\alpha(U\cap V)\times\R^n\to\varphi_\beta(U\cap V)\times\R^n$ is given by\begin{align*}
	\widetilde{\varphi}_{\alpha\beta}\left(x^1,\dots,x^n,v^1,\dots,v^n\right)=\left(y^1(x),\dots,y^n(x),\sum_{i}v^i\pdiff{y^1}{x^i}(x),\dots,\sum_{i}v^i\pdiff{y^n}{x^i}(x)\right),
	\end{align*}which is smooth. The collection $\{(\pi^{-1}(U_\alpha),\widetilde{\varphi}_\alpha)\}$ then gives a smooth atlas of charts for $TM$. Moreover, in a given chart $(\pi^{-1}(U_\alpha),\widetilde{\varphi}_\alpha)$ with co\"{o}rdinates $x=(x^1,\dots,x^n)$, the projection map $\pi$ is given by $\pi(x,v)=x$ which is clearly smooth.
\end{proof}
The tangent bundle of a manifold is an example of a more general class of smooth objects: (smooth) vector bundles.
\begin{definition}
	Let $B$ be a smooth manifold, then a \textit{rank $k$ (real) vector bundle over $B$} is a smooth manifold $E$ together with a smooth surjective map $\pi:E\to B$ such that\begin{flushleft}
		\textup{(a)} For every $b\in B$, the \textup{fiber} $E_b=\pi^{-1}(\{b\})$ has the structure of a real $k$-dimensional vector space.\\
		\textup{(b)} For every $b\in B$, there exists a neighborhood $U$ of $b$ together with a diffeomorphism $\phi:\pi^{-1}(U)\stackrel{\sim}{\to}U\times\R^k$, called a \textit{local trivialization of $E$ over $U$}, such that $p\circ\phi=\pi$, where $p:U\times\R^k\to U$ is the canonical projection map, and for each $u\in U$, the map $\left.\phi\right|_{E_u}:E_u\to\{u\}\times\R^k$ is a vector space isomorphism.
	\end{flushleft}
The space $E$ is called the \textit{total space} of the vector bundle, $B$ is called the \textit{base}, and $\pi$ is called the \textit{projection map}.
\end{definition}
\begin{definition}
	Given smooth vector bundles $\pi:E\to B$ and $\rho:F\to C$, a bundle homomorphism between them is a pair $(f,\bar{f})$ consisting of smooth maps $f:B\to C$ and $\bar{f}:E\to F$ such that the diagram
	$$
	\begin{CD}
	E @>\bar{f}>> F\\
	@V\pi VV  @VV\rho V\\
	B @>>f> C
	\end{CD}
	$$
	commutes. If $f$ and $\bar{f}$ are diffeomorphisms, we say that $(f,\bar{f})$ is a bundle isomorphism.
\end{definition}

One of the first questions one can ask about a tangent bundle $TM$ is whether or not it is trivial. To answer this question, we have to determine if $TM$ is isomorphic as a vector bundle to the trivial bundle $M\times \R^n$. To develop some familiarity with tangent bundles, we investigate the triviality of the bundles $TS^1$ and $TS^2$.
\begin{example} We argue that $TS^1$ is isomorphic to the trivial bundle $S^1\times\R$.
	We think of $S^1$ as being the unit circle in $\C$ and note that for each $x\in S^1$, the vector $ix$ is a rotation of $x$ by $\pi/2$. Defining $v_x$ to be the vector $ix$ with tail starting at $x$ gives us a continuous nonzero vector field on $S^1.$ Using this vector field, we can define a map
	\begin{align*}
	S^1\times\R&\to TS^1,\\
	(x,t)&\mapsto (x,tv_x),
 	\end{align*}
 	which is clearly a homeomorphism. Furthermore, the map sending $\{x\}\times\R$ to the tangent line at $x$ is a linear isomorphism. This tells us that our homeomorphism defines an isomorphism of $S^1$-bundles, and so $TS^1$ is trivial.
\end{example}
\begin{example}[cf. Hatcher vector bundles book]
	The tangent bundle $TS^2$ is nontrivial, which we show by constructing $TS^2$ via a clutching map $f$. Denote the closed upper hemisphere of $S^2$ by $D_+^2$, and the closed lower hemisphere by $D_-^2$. For nonzero $v_+\in T_{(0,1,0)}S^2$ in the tangent space of the north pole of $S^2$, we can define a vector field on $D_+^2$ by sliding $v_+$ down each great circle through $(0,1,0)$ to the equator, keeping the angle between $v_+$ and the great circle constant. Let $w_+$ be the rotation in $T_{(0,1,0)}S^2$ of $v_+$ by $\pi/2$, and define a similar vector field on $D^2_+$ using this $w_+$. Then $v_+,w_+$ both give trivializations of $TS^2$ on $D_+$. For nonzero $v_-\in T_{(0,-1,0)}S^2$ in the tangent space of the south pole of $S^2$ we can define a vector field on $D_-^2$ by sliding $v_-$ up along each great circle through $(0,-1,0)$ to the equator. Rotating $v_-$ by $\pi/2$ in the tangent plane, we obtain a vector field $w_-$ on $D^2_-$, and now have trivializations of $TS^2$ on $D_-^2$ given by $v_-,w_-$. 
	
	Now, extending $D_+^2,D_-^2$ and the corresponding vector fields $v_-,v_+,w_-,w_+$ by $\epsilon$-neighborhoods past the equator lets us consider these vector fields along the equatorial $S^1$. The tangent space $TS^2$ can then be recovered as a quotient of $D_+^2\times\R^2\sqcup D_-^2\times\R^2$ if we can figure out a way to identify the vector field points along the equator. The function that does this is the map $f:S^1\to\text{GL}_2(\R)$ defining the rotation needed to send the vectors $v_+,w_+$ to $v_-,w_-$. So in fact this map actually goes to $S^1$, since any rotation matrix gives a point on $S^1$. This $f$ is by definition a ``clutching function". Now, as we traverse $S^1$ and track the angle by which the pairs of vectors differ, we can see that the angle goes from 0 to $4\pi$. This tells us that the map $f$ as a map $S^1\to S^1$ has degree 2. From this we conclude that the tangent bundle $TS^2$ is nontrivial, since looking at the restriction of the transition map of the trivial bundle $S^2\times\R^2$ to the equatorial $S^1$ should be the identity map.
\end{example}
Our definition of tangent space used the well-known derivative defined on Euclidean space. It is also useful to have a notion of derivative for smooth maps between abstract manifolds, which we define now.
\begin{definition}
For  a smooth map $f\colon M\rightarrow N$ between manifolds (of dimensions $m$ and $n$, respectively, not necessarily embedded in some ambient Euclidean space), the \emph{derivative of $f$ at $x$} is a linear map $df_x\colon T_xM\rightarrow T_{f(x)}M$ which serves as the ``best linear approximation'' of $f$ in a neighborhood of $x$. This map is defined as follows: Suppose $f(x)=y$, and let $\phi\colon U\rightarrow M$ and $\psi\colon V\rightarrow N$ be local parameterizations of $x$ and $y$, respectively (where $U\subseteq \R^m$ and $V\subseteq \R^n$). We may assume without loss of generality $\phi(0)=x$ and $\psi(0)=y$. For sufficiently small $U$, we get the following commutative diagram
$$
\begin{CD}
M @>f>> N\\
@A\phi AA  @AA\psi A\\
U @>>h=\psi^{-1}\circ f\circ \phi> V
\end{CD}
$$
We now apply the derivative to the above diagram. The chain rule guarantees that commutativity is preserved.
$$
\begin{CD}
T_xM @>df_x>> T_{f(x)}N\\
@Ad\phi_0 AA  @AAd\psi_0 A\\
\R^m @>>dh_0> \R^n
\end{CD}
$$
Observe that $dh_0$ is the usual derivative between subsets of Euclidean space. Since $d\phi_0$ is a diffeomorphism, we write
\[df_x=d\psi_0\circ dh_0\circ d\phi_0^{-1}\]
\end{definition}
\begin{definition}
For a smooth map of manifolds $f\colon M\rightarrow N$, we say that $f$ is an \emph{immersion} if for every $x\in M$ we have that $df_x\colon T_xM\rightarrow T_{f(x)}N$ is an injective map.
\end{definition}
For maps between manifolds where $\dim M\leq \dim N$, being an immersion is the strongest condition we can put on the derivative. It is important to note that an immersion $f:M\to N$ may not actually be injective. When an immersion is injective and proper (proper means that the preimage of any compact set is compact), we have the following theorem, which helps us construct submanifolds.
\begin{theorem}
Let $f\colon M\rightarrow N$ be a proper injective immersion. Then $f$ maps $M$ diffeomorphically onto a submanifold of $N$.
\end{theorem}
\begin{proof}
	See Guillemin and Pollack p. 17.
\end{proof}	
If $\dim M\geq \dim N$, then we can instead ask for the derivative of a smooth map $f:M\to N$ to be surjective. In this case, we have the following theorem.
\begin{theorem}
Let $f\colon M\rightarrow N$ be a smooth map between manifolds, and let $y\in N$. Suppose that for all $x\in f^{-1}(y)$ we have $df_x$ is surjective. Then $f^{-1}(y)$ is a submanifold of $M$.
\end{theorem}
\begin{proof}
	See Guillemin and Pollack p. 21.
\end{proof}	
In the above theorem, if the condition put on $f^{-1}(y)$ holds, we call $y$ a \emph{regular value} of $f$. 

\begin{example}
Consider the smooth map $f\colon\R^{n+1}\rightarrow \R$ given by $(x_1,\dots ,x_{n+1})\mapsto x_1^2+\cdots +x_{n+1}^2$. Every nonzero element in $f^{-1}(1)$ has nonzero (ie. surjective) derivative. This is because if $a=(a_0,\dots ,a_n)$, then $df_{a}$ has Jacobian matrix $(2a_0,\dots ,2a_n)$. This linear map is surjective unless $f(a)=0$. So the preimage $f^{-1}(1)=S^n$ is a submanifold of $\R^n$.
\end{example}

\begin{definition}
Transversal intersection of submanifolds\index{transversality!of submanifolds}.
\end{definition}
\begin{figure}[H]
	\begin{center}
		\begin{pspicture}(0.5,-0.5)(3.5,3.5)			
		\pspolygon(-2,1)(0,1)(1,2)(-1,2)
		\psline(-.5,0)(-.5,1)
		\psline[linestyle=dashed,dash=1.3pt](-.5,1)(-.5,1.5)
		\pscircle*(-.5,1.5){1pt}
		\psline[linecolor=white,linewidth=1.2pt](-.47,2)(-.53,2)
		\psline(-.5,1.5)(-.5,3)	
		\rput(-3,1.5){\textup{a)}}
		
		
		\pspolygon(4,1)(4,-.25)(5.2,.75)(5.2,2)
		\psline[linecolor=white,linewidth=1.2pt](5.2,1.2)(5.2,2)
		\psline[linestyle=dashed,dash=1.3pt](5.2,1.2)(5.2,2)
		\pspolygon(3,1)(5,1)(6.2,2)(4.2,2)
		\psline[linecolor=white,linewidth=1.2pt](4.2,2)(5.2,2)
		\psline[linecolor=white,linewidth=1.2pt](4,1.83)(4.2,2)
		\psline[linestyle=dashed,dash=1.3pt](4,1.83)(4.2,2)
		\psline[linestyle=dashed,dash=1.3pt](4.2,2)(5.2,2)
		\pspolygon(4,1)(4,2.25)(5.2,3.25)(5.2,2)
		\rput(2,1.5){\textup{b)}}
		\end{pspicture}
	\end{center}
	\caption{Examples of transversally intersecting submanifolds of $\R^3$: \textup{a)} the plane $z=0$ and the line $x=y=0$ and \textup{b)} the planes $z=0$ and $y=0$.}
\end{figure}

\begin{definition}
If $M$ is a smooth manifold of dimension $m$ and $W\subseteq M$ is a smooth submanifold of dimension $k$, then the codimension\index{codimension} of $W$ in $M$ is the number $\codim W=m-k$.
\end{definition}

\begin{theorem}
If submanifolds intersect transversally, then the intersection is a submanifold.  The codimension is the sum of their codimensions.
\end{theorem}

We can now state a simplified version of the fundamental theorem of intersection theory\index{intersection theory!fundamental theorem of}.

\begin{theorem}
Let $M$ be a manifold and $W$ be a submanifold of codimension $d$ without boundary.  
\begin{enumerate}
\item There is a cohomology class $[\tau_{W}] \in H^{d}(M; \Z/2)$ represented by a cochain whose value on an embedded chain $\Delta^{d} \subset M$ which intersects $W$ transversally is the mod-two count of $\Delta^{d} \cap W$.
%\item If $S$ is a submanifold whose boundary is $V \sqcup W$ then $[\tau_{V}] = [\tau_{W}]$.
\item If $V$ and $W$ intersect transversally then $[\tau_{V}] \cup [\tau_{W}] = [\tau_{V \cap W}]$.
\end{enumerate}
\end{theorem}

\subsection{Examples using the simplified version}
\begin{example}
$\R^{2} - 0$
\end{example}
\begin{example}
$[\tau_M] = 1$.  $[\tau_{{\rm point}}]$.
\end{example}
\begin{example}
$\R^{3}$ with the z-axis removed along with either a linked or unlinked $S^{1}$.
\end{example}
\begin{example}
A two-holed torus
\end{example}
\begin{example}
$\RP^{n}$ and application to no retraction.
\end{example}

\subsection{Comparison with other approaches to cochains}

Cochains are transcendental data (a value on every chain).  But some representations are more finite.

Focus on torus example.  Both simplicial and de Rham  generally assess ``tolls'' for chains.  Both can be made ``concentrated'' as one
crosses a submanifold.  


\section{Transversality and preimage cochains}

(Adapted from treatment by Dominic Joyce in ``On manifolds with corners.'')

We operate at the interface of smooth and simplicial topology, which requires considering simplices as manifolds with corners.
Let $\R^+$ be the subset of non-negative real numbers, and let $\R^{k,\ell} = \R^k \times (\R^+)^\ell$.  A smooth map between
 subspaces of Euclidean spaces is the restriction of such between some neighborhoods of those subspaces, in which case
 the derivative exists and we can define diffeomorphism as usual.

\begin{definition}
A manifold with corners\index{manifold!with corners} is a paracompact Hausdorff space $M$ with (a maximal collection of) charts $\phi_\alpha : U_\alpha \to M$
with $U_\alpha$ open in  $\R^{0,n}$
such that the transition maps $\phi_\beta^{-1} \circ \phi_\alpha$ are diffeomorphisms between $\phi_\alpha^{-1} ({\rm Im} \phi_\beta)$
and $\phi_\beta^{-1} ({\rm Im} \phi_\alpha)$.  
\end{definition}

An important non-example for the theory of manifolds with corners is the map $\R^{0,2} \to \R^{1,1}$ 
which in polar coordinates multiplies the angle
with the $x$-axis by two, which while being smooth is not a diffeomorphism at the origin.

The tangent bundle can be defined in any of the ways in which one defines it for manifolds. 

\begin{lemma}
Every point $x$ in a manifold with corners $M$ 
has a neighborhood diffeomorphic to (a neighborhood of $0$ in) 
$\R^{k,\ell}$ for some unique $k + \ell = n$.  The number $\ell$
is called the depth\index{depth of a point} of $x$.
\end{lemma}

For example, the depth $0$ points are also known as the interior of $M$.
 
A key structure to consider for a manifold with corners is its partition into connected subspaces with fixed depth, which is sometimes
called a stratification\index{stratification}.  In the case of polytopes such as the cube this is essentially the facet structure.   A delicate issue is how to take a
``closure'' of such facets\index{stratification!facets} or strata\index{stratification!strata|see {facets}} for general manifolds with corners, for which we rely on the tangent bundle.

\begin{definition}
A codimension-$i$ facet subspace\index{stratification!facet subspace} of $T_0 \R^{k,\ell} \cong \R^{k+\ell}$ is a linear subspace in which a fixed nonempty 
subset of cardinality $i$ of the 
coordinates which correspond to coordinates in $(\R^+)^\ell$  are zero.
\end{definition}

Thus $T_0 R^{k,\ell}$ has $2^\ell - 1$ facet subspaces total, over all codimensions.

\begin{definition}
A facet subspace of $T_x M$ is the image of some facet subspace of $T_0 R^{k,\ell}$ under the derivative $D \phi$ of a chart $\phi$ about $x$.
\end{definition}

A facet subspace thus consists of vectors which are tangent to a stratum containing $x$ in its closure.  
While there is a unique minimal one, given by the image of the stratum containing $x$ itself, we need to consider the full poset of such
subspaces.
 

\begin{definition}
A boundary facet\index{stratification!boundary facet} of a manifold with corners $M$ is a maximal manifold with corners $F$ with an immersion 
$\iota_F : F \to M$, injective on the interior of $F$, such that $D\iota_F$ maps $TF$ injectively to a subbundle of facet 
subspaces of $TM$.  
\end{definition}

Example: $\Delta^n$

The only examples we need to consider are $M$ for which the facet immersions $\iota_F$ are embeddings, as happens for polytopes.
Such $M$ are called manifolds with embedded corners\index{manifold!with embedded corners}.  
The ``teardrop'' subspace of $\R^2$ (homeomorphic to a disk, with a single corner) is  a typical example of a manifold with corners
which does not have embedded corners.

Constructions such as products and (good) intersections of manifolds with corners are rich because of the interplay of the stratifications in 
a manifold-with-corners structure on the result.  Consider, for example, a generic linear intersection between a solid cube and the 
boundary of a tetrahedron.
We will not take the opportunity to enjoy working through this, because the submanifolds-with-corners we need have dimension
zero or one, and thus have at most boundaries.


Transversality more generally.

\begin{definition}
Let $M$ be a manifold and $f : W \to M$ an immersion. Then $C^{\pitchfork W}_{*}(M)$ is the subset of $C_{*}(M)$ generated by the chains whose image is transversal to $f$. More generally, if $S$ is a  finite set of immersed submanifolds of $M$ then $C^{\pitchfork S}_{*}$\index{$C_{\pitchfork S}^*$} is generated by the chains whose image is transversal to each $W \in S$.
\end{definition}

\begin{theorem}\label{T:qi}
For any finite collection $S$ of submanifolds of $M$, the inclusion $C^{\pitchfork S}_{*}(M) \hookrightarrow C_{*}(M)$ admits a chain deformation retraction.
\end{theorem}


\begin{proof} We wish to construct a map $j:C_{*}(M) \rightarrow C^{\pitchfork S}_{*}(M)$ such that $j$ is the identity on $C^{\pitchfork S}_{*}(M)$ and 
$i \circ j$ is chain homotopic to the identity on $C_{*}(M)$. Given $\sigma:\Delta^d \rightarrow M$, we can cover its image with a finite number of 
open sets $\{U_i\}$ in $M$ diffeomorphic to open subsets of $\R^k$, and we can choose these so that on the overlap the transition from one 
diffeomorphism to another is smooth. We proceed inductively: \\If $e^0_k \subset U_k$ is not transversal to $S$, then $e^0_k \in S$ and by 
Sard's Theorem we may choose a path $\gamma_k$ in $U_k$ from $e^0_k$ to a point not in $S$. If $e^0_k \pitchfork S$, then $\gamma_k$ is the 
constant map. Using the consistent cover of $\sigma$, we may extend this to a map $\bar{\sigma}:\Delta^d \rightarrow M$ such that $\bar{\sigma}$ 
on the 0-cells is defined by the endpoints of the $\gamma_k$'s. This gives a map 
$\left(\Delta^d \times \{0,1\} \right)\cup \left(\{e^0_k\} \times I \right)\rightarrow M$, and simplicial complexes have the homotopy extension property, 
so can extend this to a homotopy $F: \Delta^d \times I \rightarrow M$. Then by Theorem ??, there exists a smooth homotopy $F'$ which 
agrees with $F$ on the 0-cells and $\sigma'=\left.F'\right|_{\Delta^d \times \{1\}}$ is transversal to $S$.\\
Now assume that we have a homotopy $\partial$G on the cells of dimension $<q$. If $e^q_k $ is not transversal to $S$, then we........  
[[Use relative transversality
extension.]]


We define $j(\sigma) = \sigma'$ and since $\left.j\right|_{C^{\pitchfork S}_{*}(M)}=id$ and the two chains are always homotopic, $C^{\pitchfork S}_{*}(M) \hookrightarrow C_{*}(M)$ is a chain deformation retraction and their homology groups are isomorphic.
\end{proof}

We now define our main objects of study.
\begin{definition}
If $W \in S$ then $\tau_{W} \in C^{*}_{\pitchfork S}(M)$ is the cochain defined by $\tau_W (\sigma) = \# P,$  where $P$ is
the  zero-manifold defined as the following pullback
$$
\begin{CD}
P @>>> W\\
@VVV  @VfVV\\
\Delta^n @>\sigma>> M.
\end{CD}
$$
The cardinality $\# P$ is to be taken as mod-two unless $W$  is co-oriented, that is, its normal bundle is oriented.  This occurs
for example when it and $M$ are both oriented.  Signs are then given by the question of agreement of the co-orientation of $P$, 
which could be a local orientation in $\Delta^n$ at each point, with the standard orientation of $\Delta^n$ given by say the differences between vertices labeled by $1, \cdots, n$ and vertex $0$ (or the barycenter) serving as a basis.
\end{definition}

We call these Thom classes\index{Thom class}, because in the standard development they are (the pushforwards of)  Thom classes of the normal bundles.  But we will use 
them to understand both pushforward and Thom classes, rather than the other way around.

\begin{theorem}
If $\partial V = W$ then $\delta \tau_{V} = \tau_{W} \in C^{*}_{\pitchfork \{V, W\}}$.  In particular, if $V$ has no boundary than $\tau_{V}$ is a cocycle.
\end{theorem}

\begin{proof}[Sketch, for now]
This comes down to the classification of one-manifolds.  Consider $\sigma : \Delta^{n+1} \to M$.  
Look at pull back of $W$ through $\sigma$ to get a one-manifold with boundary.  Some boundary points are on $\partial \Delta^{n+1}$ - 
counting those gives 
 $\delta \tau_V (\sigma)  = \tau_V (\partial \sigma)$.  Other boundary points are in the interior, which come from the preimage of 
$\partial V = W$, and thus correspond to $\tau_{W} (\sigma)$.
\end{proof}

\begin{figure}[H]
	\begin{center}
		\begin{pspicture}(0.5,-0.5)(3.5,3.5)			
		\pspolygon[linewidth=1.2pt](0,0)(4,0)(2,3.46410161514)
		\psarc[linecolor=orange!90,linewidth=1.2pt](0.66666,1.15470053838){1.5em}{-120}{60}
		\pscurve[linecolor=orange!90,linewidth=1.2pt](1.33333,2.3094017676)(1.7,2)(1.9,2.1)(2.45,1.7)
		\psdot[linecolor=red!90,fillcolor=red!90](0.41,0.7)
		\psdot[linecolor=red!90,fillcolor=red!90](0.93,1.6)
		\psdot[linecolor=red!90,fillcolor=red!90](1.3333,2.3094)
		\psdot[linecolor=blue!70!violet!60,fillcolor=red!50](2.44,1.7)
		\psline[linecolor=orange!90,linewidth=1.2pt](2,1.5)(3,1)
		\psdot[linecolor=blue!80!violet!60,fillcolor=red!50](2,1.5)
		\psdot[linecolor=blue!80!violet!60,fillcolor=red!50](3,1)
		\psccurve[linecolor=orange!90,fillcolor=red!90](1.5,0.5)(1.8,0.35)(2,0.5)(2.2,0.35)(2.5,0.5)(2,1)
		\end{pspicture}
	\end{center}
\caption{Example in the case $n=1$ with $\sigma^{-1}(V)$ in orange, points in $\left(\left.\sigma\right|_{\partial\Delta^2}\right)^{-1}(V)$ marked in red, and those in $\sigma^{-1}(\partial V)$ marked in blue.}
\end{figure}

By abuse, we denote by $[\tau_{W}]$ the corresponding singular cohomology class under the isomorphism established in Theorem~\ref{T:qi}.

\begin{corollary}
If $\partial V = W \sqcup W'$ then $[\tau_W] = [\tau_{W'}]$.
\end{corollary}

This theorem and its
corollary recovers a classical view of cohomology, as represented by 
submanifolds with the relation defined by submanifolds with boundary.  But in fact
not all cocycles are represented in this way, which was a famous question of Steenrod addressed by Renee Thom.  
Our theory allows for linear combinations
of submanifolds, and even manifolds with corners below, in which case all cohomology is representable.  
The submanifold point of view led Thom
to initiate cobordism theory, which gives generalized cohomology theories.



Induced maps are a basic ingredient of cohomology theory.  
In the case of Thom classes, they are geometrically defined on the cochain level.  The
proof of the following is immediate.

\begin{proposition}\label{pullback}
If $f : M \to N$ is transverse to $W \subset N$ then $f^{\#}(\tau_{W}) = \tau_{f^{-1} W}$.
\end{proposition}

We summarize a yoga to find complete presentations of homology of manifolds.
First, one calculates the homology up to isomorphism, using cellular structures, exact sequences or other tools
(such as spectral sequences, eventually).  Then one finds  cycles to produce homology, either through singular chains
or implicitly through compact submanifolds without boundary.  [** Need to discuss this further somewhere **]   One
also finds submanifolds, not necessarily compact, of complimentary dimension to define cohomology.  Then one
calculates the pairing, and checks whether the homology calculated is accounted for, and goes back to produce more homology
and cohomology until it is.


More examples



\section{Intersection and cup product}

In this section we prove the following.

\begin{theorem}
If $V$ and $W$ intersect transversally, and the intersection is given the co-orientation... then $[\tau_{V}] \smile [\tau_W] = [\tau_{V \cap W}]$.
\end{theorem}

Recall that the cup product is essentially induced by the diagonal map $X \overset{\Delta}{\to} X \times X$.  One standard
approach to the cohomology of a product $X \times Y$ is developed through the external cup product, namely $C^*(X) \otimes C^*(Y)$
maps to $C^*(X \times Y)$ by sending $\alpha  \otimes \beta$ to $p^* \alpha \smile q^*\beta$, where $p$ is the projection from 
$X \times Y$ to $X$ and $q$ is the projection to $Y$.  This map is an isomorphism with field coefficients or when 
one of $X$ and $Y$ has torsion-free cohomology -- see Theorem~3.16 of Hatcher's book **, which establishes this when $X$ and
$Y$ are CW-complexes.

\begin{proposition}\label{externalcup}
The external cup product sends $[\tau_V] \otimes [\tau_W]$ to $[\tau_{V \times W}]$
\end{proposition}

\begin{proof}[Proof of the Fundamental Theorem, based on Proposition~\ref{externalcup}]
Suppose that $V\pitchfork W$ and suppose we have $(x,x)\in\Delta(X)\cap V\times W$, i.e. that $x\in V\cap W$. Recall that $T_{(x,x)}M\times M\cong T_xM\oplus T_xM$ and suppose that, under this identification, we have $(\alpha,\beta)\in T_{(x,x)}M\times M$. Since $V\pitchfork W$, we may write $(\alpha,\beta)=(v_1+w_1,v_2,w_2)$ for some $v_i\in T_xV$ and $w_i\in T_xW$ but, then we have\begin{align*}
(\alpha,\beta)=(v_2+w_1,v_2+w_1)+(v_1-v_2,w_2-w_1)
\end{align*}and we have that $(v_2+w_1,v_2+w_1)\in\mathrm{im}\,\mathrm{d}\Delta_x$, since $\mathrm{d}:T_xM\to T_xM\oplus T_xM$ is again the diagonal map, and $(v_1-v_2,w_2-w_1)\in T_xV\oplus T_xW\cong T_{(x,x)}V\times W$ so we have that $\Delta\pitchfork V\times W$. Now consider the compositions $M\stackrel{\Delta}{\to}M\times M\stackrel{p_1,p_2}{\longrightarrow}M$, where $p_i$ is the projection onto the $i^\text{th}$ factor of $M$, and note that $\mathrm{d}(p_i)_x:T_xM\oplus T_xM\to T_xM$ is itself the projection map onto the $i^\text{th}$ factor of $T_xM$ so we automatically have $p_1\pitchfork V$ and $p_2\pitchfork W$. By our above remarks, we have that $p_1^*[\tau_V]=\left[\tau_{p_1^{-1}(V)}\right]=[\tau_{V\times M}]$ and $p_2^*[\tau_W]=\left[\tau_{p_2^{-1}(W)}\right]=[\tau_{M\times W}]$ and $\Delta\pitchfork V\times W$ so we obtain\begin{align*}
[\tau_V]\smile[\tau_W]&=\Delta^*(p_1^*[\tau_V]\smile p_2^*[\tau_W])\\&=\Delta^*([\tau_{V\times M}]\smile[\tau_{M\times W}])\\&=\Delta^*[\tau_{V\times W}]\\&=\left[\tau_{\Delta^{-1}(V\times W)}\right]\\&=[\tau_{V\cap W}],
\end{align*}as desired, by Proposition~\ref{pullback} and Proposition~\ref{externalcup}.
\end{proof}


In order to prove Proposition~\ref{externalcup} we need some chain-level analysis of the K\"unneth theorem, starting with 
a subdivision of simplices which we approach through the language of partially ordered sets.

\begin{definition}
Let ${\rm Ord}(I,J)$ be the set of order preserving maps between partially ordered sets $I$ and $J$.  Let $\I$ denote the unit interval,
with partial ordering $\leq$.  An order preserving map from $\I$ to some $J$ will be a point in  $\I^J$.  When $J$ is finite, we topologize
${\rm Ord}(\I, J)$ as a subspace of $\I^J$.
\end{definition}

\begin{example}
Let $[n] = \{0, \ldots, n\}$, with the standard ordering.  Then ${\rm Ord}(\I, [n]) \cong \Delta^n$.  Thus  ${\rm Ord}(I,J)$ for any $J$
is (the realization of) a simplicial complex, called the order complex of $J$, with $n$-simplices which correspond to flags(?)
of length $n + 1$ in $J$.
\end{example}

Because ${\rm Ord}(I,J \times K) \cong {\rm Ord}(I,J) \times {\rm Ord}(I,K)$ we can use ${\rm Ord}(\I,[n] \times [m])$
as a model for $\Delta^n \times \Delta^m$.   

\begin{exercise}
Show that the number of maximal flags(?) in $[n] \times [m]$, which gives the number of
$n + m$ simplices in the order complex ${\rm Ord}(\I, [n] \times [m])$ is equal to the number of shuffles of $[n]$ and $[m]$.
\end{exercise}

\begin{definition}
Let $k_{n,m} \in C_*(\Delta^n \times \Delta^m)$ be the sum of the $n + m$ simplices in the order complex ${\rm Ord}(\I, [n] \times [m])$
with signs given by...  

Let 
\end{definition}


\begin{proof}[Proof of Proposition~\ref{externalcup}]
Recall K\"unneth theorem that  $C_*(X) \otimes C_*(Y) \to C_*(X \times Y)$ given by (subdividing) products of simplices in $X$ and $Y$ is an
isomorphism on homology.  

Then show that $p^* \tau_V \cup q^* \tau_W$ and $\tau_{V \times W}$ agree on this subcomplex, which is has an immediate
geometric proof.
\end{proof}

\section{Examples}

Recall from Theorem~3.26 of Hatcher** the notion of fundamental class of of manifold, which  is much more common to 
develop than Thom classes.  
But we see that Thom classes are just as natural if not more so.


\begin{proposition}\label{top_cohomology}
The value of a Thom cocycle associated to a zero-dimensional submanifold on the fundamental class of a compact manifold without
boundary  is given by 
the signed count of the submanifold.
\end{proposition}

\begin{proof}
Use characterization of fundamental class as restricting to generators of local homology to reduce to the pair $(\R^n, \R^n - 0)$,
which is easy to analyze.
\end{proof}

Remark that this is immediate if the manifold is triangulated.

\begin{proposition}
The value of a Thom class $[\tau_W]$ on  a fundamental class of a submanifold is given by signed intersection with 
(or more generally preimage of) $W$
\end{proposition}

\begin{proof}
Use naturality of evaluating cohomology on homology, and then apply Proposition~\ref{top_cohomology}
\end{proof}


\subsection{Projective spaces}\index{projective space}

\subsection{Grassmannians}\index{Grassmanian}
\begin{definition}
	Let $V$ be a real vector space and $k\geq 1$ an integer. The \textit{Grassmannian of real $k$-planes in $V$} is the set\begin{align*}
	\Gr(k,V)=\{W\subseteq V:\text{$W$ is a linear subspace of $V$ and $\dim W=k$}\}.
	\end{align*}
	In the special case that $V=\R^n$, we define $\Gr(k,n)=\Gr(k,\R^n)$.
\end{definition}
\begin{definition}
	If, instead, $V$ is a \textit{complex} vector space, we may define the \textit{Grassmannian of complex $k$-planes in $V$} by\begin{align*}
	\Gr^\C(k,V)=\{W\subseteq V:\text{$W$ is a $\C$-linear subspace of $V$ and $\dim W=k$}\}.
	\end{align*}
	By way of analogy with the real case, we define $\Gr^\C(k,n)=\Gr^\C(k,\C^n)$.
\end{definition}
\begin{example}
	In the particular case that $k=1$, we have $\Gr(1,n)=\RP^n$ and $\Gr^\C(1,n)=\CP^n$.
\end{example}
\begin{proposition}
	For all $k\geq 1$ and all $n\geq 0$, $\Gr(k,n)$ and $\Gr^\C(k,n)$ are smooth manifolds.
\end{proposition}
\begin{proof}
	Fill this in.
\end{proof}
\begin{remark}
	In fact, $\Gr^\C(k,n)$ is a complex manifold of (complex) dimension $(n-k)k$, which is to say that $\Gr^\C(k,n)$ admits a smooth atlas of charts $(U_\alpha,\varphi_\alpha)$ such that the transition functions $\varphi_{\alpha\beta}=\left.\varphi_\beta\circ\varphi_{\alpha}^{-1}\right|_{\Omega_{\alpha\beta}}$, where $\Omega_{\alpha\beta}=\varphi_\alpha(U_\alpha\cap U_\beta)$, are holomorphic in each variable as maps $\Omega_{\alpha\beta}\to\C^{(n-k)k}$.
\end{remark}

\subsection{Complements of submanifolds which are boundaries}

\subsection{Configuration spaces}\index{configuration space}

\section{Applications: wrong-way maps\index{wrong-way map}, duality\index{duality}, suspensions\index{suspension} and Thom isomorphism\index{Thom isomorphism}}

\section{Looking forward: characteristic classes\index{characteristic class}, loop spaces\index{loop space}, and cobordism\index{cobordism}.}

\nocite{*}
\bibliographystyle{plain}
\bibliography{bib}
\printindex
\end{document}

 
