\documentclass{amsart}          % Nicer than default article style:  less
                                % flashy headings, etc.

%\usepackage{amsmath,amsthm}     % Handy math stuff, theorem environments.
\usepackage{amssymb, amscd}            % Fancy math symbols.


\newcommand{\Z}{{\mathbb{Z}}}

\textwidth = 6.5 in
\textheight = 8.5 in
\oddsidemargin = 0.0 in
\evensidemargin = 0.0 in
%\topmargin = 0.0 in
%\headheight = 0.0 in
%\headsep = 0.0 in
\parskip = 0.2in
\parindent = 0.0in

\newtheorem{theorem}{Theorem}
\newtheorem{corollary}[theorem]{Corollary}
\newtheorem{proposition}[theorem]{Proposition}
\newtheorem{lemma}[theorem]{Lemma}
\newtheorem{definition}[theorem]{Definition}
\newtheorem{remark}[theorem]{Remark}
\newtheorem{example}{Example}

\newcommand{\R}{\mathbb R}
\newcommand{\Q}{\mathbb Q}
\newcommand{\colim}{{\rm colim}}
\newcommand{\C}{\mathbb C}
\newcommand{\D}{\mathcal D}
\newcommand{\E}{\mathcal E}
%\newcommand{\ker}{{\rm ker}\;}
\newcommand{\coker}{{\rm coker}\;}
\newcommand{\ext}{{\rm Ext}}
\newcommand{\injext}{{\rm InjExt}}
\newcommand{\Hom}{{\rm Hom}}

\begin{document}

\title{Elementary Intersection Theory}
%\author{Dev Sinha}
\maketitle

\section{Introduction}

We present a treatment of elementary intersection theory in algebraic topology.  
That is, we aim to show how one can  
define cochains through counting intersection with or more generally preimages of
submanifolds.  Moreover, the ``fundamental theorem''  is that the cup product of the associated
 cohomology classes is represented by the (transversal) intersection of submanifolds.  
 Students are often told that cup product is named so because of this 
 fact, but we feel it is given relatively short shrift.  It is usually proved after development of duality or a Thom isomorphism theorem.  
 We would like to put the representation of cohomology by submanifolds at the front and center of our treatment and from that deduce, 
 or at least interpret, these related isomorphisms.

Unlike chains, singular cochains tend to be more transcendental objects.  From a pedagogical  viewpoint, one can often express cycles as in terms of explicit chains, which is basically never the case for singular cocyles.  Our work remedies this discrepancy, and  allows for geometric cochain-level understanding of a number of topics in algebraic topology including duality, Thom isomorphisms, cohomology of mapping spaces
etc. 


\section{Basic definitions, and a simplified version of the fundamental theorem}

Before developing the background in differential topology needed, we illustrate the technique by using simplified definitions.

\begin{definition}
	A subset $M$ of Euclidean space $\R^n$ is called a (smooth) $k$-dimensional manifold if it is locally diffeomorphic to $\R^k$; that is, for every point in $M$ there is a neighborhood of that point which is diffeomorphic to $\R^k$.  
\end{definition}

We often omit the word ``smooth," and that qualifier will be assumed throughout these notes unless otherwise stated. There are a number of different ways to describe manifolds, and in particular we will find it useful to describe them via ``parametrizations" and via solution sets to various equations. The latter definition takes more machinery to justify, but we will illustrate an early example for motivation.

Establishing local diffeomorphisms which make a set $M\subset \R^n$ into a manifold does more than simply establishing that $M$ is a manifold, but provides machinery allowing us to establish other properties of $M$. We call a map $\phi: \R^k \rightarrow M$ which is a diffeomorphism onto its image $U$ a \textit{parametrization} of its image. We could equivalently replace $\R^k$ with an open subset of $\R^k$ if we desired. The inverse map $\phi^{-1}:U\rightarrow \R^k$ is called a \textit{coordinate system} on $U$, and the pairing $(U, \phi)$ is called a \textit{chart}. The existence of a collection of charts $\{ (U_\alpha, \varphi_\alpha) | \alpha \in A\}$ such that $\bigcup_{\alpha\in A}U_\alpha = M$ is equivalent to the condition that $M$ is a manifold. Such a collection is called an \textit{atlas}. In the case where $M$ is compact, every atlas can be reduced to finitely many charts. We can use parametrizations and charts to prove that a space is a manifold by covering it with a collection of diffeomorphisms from $\R^k$.

\begin{example}
	$S^1 = \{(x,y)\in \R^2 | x^2 + y^2=1\}$ is a manifold with dimension $1$. \\
	
	We can exhibit four parametrizing maps from $(-1,1)$ into $S^1$. Given $x \in (-1,1)$, the map onto the upper open semicircle given by $x\mapsto(x, \sqrt{1-x^2})$ is smooth. We can define a similar map on to the lower open semicircle given by $x \mapsto (x, 1-\sqrt{1-x^2})$. These two maps alone \textit{do not} show that $S^1$ is a manifold: The points at $(1,0)$ and $(-1,0)$ are not in the image of either map. Two similar maps sending $(-1,1)$ smoothly onto the left and right open semicircles completes a parametrization of $S^1$.  
	
	[image here]
	
	Since $(-1,1)$ is diffeomorphic to $\R$ this proves that $S^1$ is a manifold and establishes that the dimension of $S^1$ is 1. \\
\end{example} 

\begin{example}
	$\R P^n$ can be covered by precisely $n+1$ charts. Consider $\R P^n$ as $\R^{n+1} \backslash \{0\}$ under the equivalence relation $(x_1,..., x_{n+1})\sim (ax_1,...,ax_{n+1})$ for $a \in \R$. Define $U_i$ as the subset of $\R P^n$ given by all elements whose $i$th coordinate is nonzero. In $U_i$, each element has a unique representative such that $x_i=1$, written as $[x_0,...,x_{i-1},1,x_{i+1},...x_{n+1}]$. This gives us a way to embed the charts $U_i$ as copies of $\R^n$ into $\R^{n+1}$ by setting the $i$th component equal to 1. This establishes that $\R P^n$ is an $n$-manifold. A visualization of the subspaces of $\R P^2$  which result from this process, depicted on the well-known cellular structure, is shown below.
	
	[image here]  
	
\end{example}

\begin{example}
	The set $S$ of all points in $\R^2$ given by ${(x,y)|xy=\epsilon, \epsilon\neq 0}$ is a $1$-manifold parametrized by the two maps $x \mapsto \frac{\epsilon}{x}$ for $x\in (0,\infty)$ and $x \in (-\infty,0)$. The solution set when $\epsilon = 0$ is not a manifold, however, because no neighborhood containing $(0,0)$ is locally diffeomorphic to $\R$: Removing the point $(0,0)$ from any such neighborhood results in four connected components, not two.  
	
	[image here]
\end{example}

\begin{definition}
	A subset $N\subseteq M\subseteq \R^n$ is called a submanifold of $M$ if $N$ is also a manifold.
\end{definition}

Note, in particular, that if $N$ is a submanifold of $M$ that $N$ and $M$ do not need to have the same dimension, and it is generally more interesting when they do not: it is a fact that if $N$ is a compact submanifold of a connected manifold $M$ with the same dimension, then $N=M$. 

We next present an initial discussion of tangent spaces, for manifolds embedded in Euclidean space. Tangent spaces are fundamental to many of the definitions related to intersection theory that we will see later.

Let $M$ be a smooth submanifold of $\R^n$. 
\begin{definition}
Let $x\in M$ be a point. Fix a local parametrization $\phi:U\to X$ mapping an open subset $U\subseteq\R^k$ to a neighbourhood of $x$. We take $\phi(0)=x$ without loss of generality. A \emph{tangent vector at $x$} is an element of the vector space $T_x M$, which is defined to be the image of the map $d\phi_0:U\to X$.
\end{definition}
We note that this definition of $T_xM$ makes sense, since $U$ and $X$ are subsets of the vector spaces $\R^k$ and $\R^n$ respectively, and $d\phi$ is a linear map.
\begin{definition}
Our first definition of \emph{the tangent bundle of $M$} is given as the following subset of $M\times\R^n$:
\[
TM:=
\{
(x,v)~|~x\in M,~v\in T_x M
\}.
\]
\end{definition}
One of the first things we can ask about a tangent bundle $TM$ is whether or not it is trivial. To answer this question, we have to determine if $TM$ is isomorphic as a vector bundle to the trivial bundle $M\times \R^n$. To develop some familiarity with tangent bundles, we investigate the triviality of the bundles $TS^1$ and $TS^2$.
\begin{example} We argue that $TS^1$ is isomorphic to the trivial bundle $S^1\times\R$.
	We think of $S^1$ as being the unit circle in $\C$ and note that for each $x\in S^1$, the vector $ix$ is a rotation of $x$ by $\pi/2$. Defining $v_x$ to be the vector $ix$ with tail starting at $x$ gives us a continuous nonzero vector field on $S^1.$ Using this vector field, we can define a map
	\begin{align*}
	S^1\times\R&\to TS^1,\\
	(x,t)&\mapsto (x,tv_x),
 	\end{align*}
 	which is clearly a homeomorphism. Furthermore, the map sending $\{x\}\times\R$ to the tangent line at $x$ is a linear isomorphism. This tells us that our homeomorphism defines an isomorphism of $S^1$-bundles, and so $TS^1$ is trivial.
\end{example}
\begin{example}[cf. Hatcher vector bundles book]
	The tangent bundle $TS^2$ is nontrivial, which we show by constructing $TS^2$ via a clutching map $f$. Denote the closed upper hemisphere of $S^2$ by $D_+^2$, and the closed lower hemisphere by $D_-^2$. For nonzero $v_+\in T_{(0,1,0)}S^2$ in the tangent space of the north pole of $S^2$, we can define a vector field on $D_+^2$ by sliding $v_+$ down each great circle through $(0,1,0)$ to the equator, keeping the angle between $v_+$ and the great circle constant. Let $w_+$ be the rotation in $T_{(0,1,0)}S^2$ of $v_+$ by $\pi/2$, and define a similar vector field on $D^2_+$ using this $w_+$. Then $v_+,w_+$ both give trivializations of $TS^2$ on $D_+$. For nonzero $v_-\in T_{(0,-1,0)}S^2$ in the tangent space of the south pole of $S^2$ we can define a vector field on $D_-^2$ by sliding $v_-$ up along each great circle through $(0,-1,0)$ to the equator. Rotating $v_-$ by $\pi/2$ in the tangent plane, we obtain a vector field $w_-$ on $D^2_-$, and now have trivializations of $TS^2$ on $D_-^2$ given by $v_-,w_-$. 
	
	Now, extending $D_+^2,D_-^2$ and the corresponding vector fields $v_-,v_+,w_-,w_+$ by $\epsilon$-neighborhoods past the equator lets us consider these vector fields along the equatorial $S^1$. The tangent space $TS^2$ can then be recovered as a quotient of $D_+^2\times\R^2\sqcup D_-^2\times\R^2$ if we can figure out a way to identify the vector field points along the equator. The function that does this is the map $f:S^1\to\text{GL}_2(\R)$ defining the rotation needed to send the vectors $v_+,w_+$ to $v_-,w_-$. So in fact this map actually goes to $S^1$, since any rotation matrix gives a point on $S^1$. This $f$ is by definition a ``clutching function". Now, as we traverse $S^1$ and track the angle by which the pairs of vectors differ, we can see that the angle goes from 0 to $4\pi$. This tells us that the map $f$ as a map $S^1\to S^1$ has degree 2. From this we conclude that the tangent bundle $TS^2$ is nontrivial, since looking at the restriction of the transition map of the trivial bundle $S^2\times\R^2$ to the equatorial $S^1$ should be the identity map.
\end{example}
Our definition of tangent space used the well-known derivative defined on Euclidean space. It is also useful to have a notion of derivative for smooth maps between abstract manifolds, which we define now.
\begin{definition}
For  a smooth map $f\colon M\rightarrow N$ between manifolds (of dimensions $m$ and $n$, respectively, not necessarily embedded in some ambient Euclidean space), the \emph{derivative of $f$ at $x$} is a linear map $df_x\colon T_xM\rightarrow T_{f(x)}M$ which serves as the ``best linear approximation'' of $f$ in a neighborhood of $x$. This map is defined as follows: Suppose $f(x)=y$, and let $\phi\colon U\rightarrow M$ and $\psi\colon V\rightarrow N$ be local parameterizations of $x$ and $y$, respectively (where $U\subseteq \R^m$ and $V\subseteq \R^n$). We may assume without loss of generality $\phi(0)=x$ and $\psi(0)=y$. For sufficiently small $U$, we get the following commutative diagram
$$
\begin{CD}
M @>f>> N\\
@A\phi AA  @AA\psi A\\
U @>>h=\psi^{-1}\circ f\circ \phi> V
\end{CD}
$$
We now apply the derivative to the above diagram. The chain rule guarantees that commutativity is preserved.
$$
\begin{CD}
T_xM @>df_x>> T_{f(x)}N\\
@Ad\phi_0 AA  @AAd\psi_0 A\\
\R^m @>>dh_0> \R^n
\end{CD}
$$
Observe that $dh_0$ is the usual derivative between subsets of Euclidean space. Since $d\phi_0$ is a diffeomorphism, we write
\[df_x=d\psi_0\circ dh_0\circ d\phi_0^{-1}\]
\end{definition}
\begin{definition}
For a smooth map of manifolds $f\colon M\rightarrow N$, we say that $f$ is an \emph{immersion} if for every $x\in M$ we have that $df_x\colon T_xM\rightarrow T_{f(x)}N$ is an injective map.
\end{definition}
For maps between manifolds where $\dim M\leq \dim N$, being an immersion is the strongest condition we can put on the derivative. It is important to note that an immersion $f:M\to N$ may not actually be injective. When an immersion is injective and proper (proper means that the preimage of any compact set is compact), we have the following theorem, which helps us construct submanifolds.
\begin{theorem}
Let $f\colon M\rightarrow N$ be a proper injective immersion. Then $f$ maps $M$ diffeomorphically onto a submanifold of $N$.
\end{theorem}
\begin{proof}
	See Guillemin and Pollack p. 17.
\end{proof}	
If $\dim M\geq \dim N$, then we can instead ask for the derivative of a smooth map $f:M\to N$ to be surjective. In this case, we have the following theorem.
\begin{theorem}
Let $f\colon M\rightarrow N$ be a smooth map between manifolds, and let $y\in N$. Suppose that for all $x\in f^{-1}(y)$ we have $df_x$ is surjective. Then $f^{-1}(y)$ is a submanifold of $M$.
\end{theorem}
\begin{proof}
	See Guillemin and Pollack p. 21.
\end{proof}	
In the above theorem, if the condition put on $f^{-1}(y)$ holds, we call $y$ a \emph{regular value} of $f$. 

\begin{example}
Consider the smooth map $f\colon\R^{n+1}\rightarrow \R$ given by $(x_1,\dots ,x_{n+1})\mapsto x_1^2+\cdots +x_{n+1}^2$. Every nonzero element in $f^{-1}(1)$ has nonzero (ie. surjective) derivative. This is because if $a=(a_0,\dots ,a_n)$, then $df_{a}$ has Jacobian matrix $(2a_0,\dots ,2a_n)$. This linear map is surjective unless $f(a)=0$. So the preimage $f^{-1}(1)=S^n$ is a submanifold of $\R^n$.
\end{example}
\begin{definition}
Transversal intersection of submanifolds.
\end{definition}

\begin{definition}
Codimension.
\end{definition}

\begin{theorem}
If submanifolds intersect transversally, then the intersection is a submanifold.  The codimension is the sum of their codimensions.
\end{theorem}

We can now state a simplified version of the fundamental theorem of intersection theory.

\begin{theorem}
Let $M$ be a manifold and $W$ be a submanifold of codimension $d$ without boundary.  
\begin{enumerate}
\item There is a cohomology class $[\tau_{W}] \in H^{d}(M; \Z/2)$ represented by a cochain whose value on an embedded chain $\Delta^{d} \subset M$ which intersects $W$ transversally is the mod-two count of $\Delta^{d} \cap W$.
\item If $S$ is a submanifold whose boundary is $V \bigsqcup W$ then $[\tau_{V}] = [\tau_{W}]$.
\item If $V$ and $W$ intersect transversally then $[\tau_{V}] \cup [\tau_{W}] = [\tau_{V \cap W}]$.
\end{enumerate}
\end{theorem}

\subsection{Examples using the simplified version}
\begin{example}
$\R^{2} - 0$
\end{example}
\begin{example}
$[\tau_M] = 1$.  $[\tau_{{\rm point}}]$.
\end{example}
\begin{example}
$\R^{3}$ with the z-axis removed along with either a linked or unlinked $S^{1}$.
\end{example}
\begin{example}
A two-holed torus
\end{example}
\begin{example}
$\R P^{n}$ and application to no retraction.
\end{example}

\subsection{Comparison with other approaches to cochains}

Cochains are transcendental data (a value on every chain).  But some representations are more finite.

Focus on torus example.  Both simplicial and de Rham  generally assess ``tolls'' for chains.  Both can be made ``concentrated'' as one
crosses a submanifold.  


\section{Transversality and intersection cochains}

\begin{definition}
Transversality more generally.
\end{definition}

\begin{definition}
Let $M$ be a manifold and $f : W \to M$ an immersion. Then $C^{\pitchfork W}_{*}(M)$ is the subset of $C_{*}(M)$ generated by the chains whose image is transversal to $f$. More generally, if $S$ is a  finite set of immersed submanifolds of $M$ then $C^{\pitchfork S}_{*}$ is generated by the chains whose image is transversal to each $W \in S$.
\end{definition}

\begin{theorem}\label{T:qi}
For any finite collection $S$ of submanifolds of $M$, the inclusion $C^{\pitchfork S}_{*}(M) \hookrightarrow C_{*}(M)$ admits a chain deformation retraction.
\end{theorem}


\begin{proof} We wish to construct a map $j:C_{*}(M) \rightarrow C^{\pitchfork S}_{*}(M)$ such that $j$ is the identity on $C^{\pitchfork S}_{*}(M)$ and 
$i \circ j$ is chain homotopic to the identity on $C_{*}(M)$. Given $\sigma:\Delta^d \rightarrow M$, we can cover its image with a finite number of 
open sets $\{U_i\}$ in $M$ diffeomorphic to open subsets of $\R^k$, and we can choose these so that on the overlap the transition from one 
diffeomorphism to another is smooth. We proceed inductively: \\If $e^0_k \subset U_k$ is not transversal to $S$, then $e^0_k \in S$ and by 
Sard's Theorem we may choose a path $\gamma_k$ in $U_k$ from $e^0_k$ to a point not in $S$. If $e^0_k \pitchfork S$, then $\gamma_k$ is the 
constant map. Using the consistent cover of $\sigma$, we may extend this to a map $\bar{\sigma}:\Delta^d \rightarrow M$ such that $\bar{\sigma}$ 
on the 0-cells is defined by the endpoints of the $\gamma_k$'s. This gives a map 
$\left(\Delta^d \times \{0,1\} \right)\cup \left(\{e^0_k\} \times I \right)\rightarrow M$, and simplicial complexes have the homotopy extension property, 
so can extend this to a homotopy $F: \Delta^d \times I \rightarrow M$. Then by Theorem~\ref{T:qr}, there exists a smooth homotopy $F'$ which 
agrees with $F$ on the 0-cells and $\sigma'=\left.F'\right|_{\Delta^d \times \{1\}}$ is transversal to $S$.\\
Now assume that we have a homotopy $\partial$G on the cells of dimension $<q$. If $e^q_k $ is not transversal to $S$, then we........  
[[Use relative transversality
extension.]]


We define $j(\sigma) = \sigma'$ and since $\left.j\right|_{C^{\pitchfork S}_{*}(M)}=id$ and the two chains are always homotopic, $C^{\pitchfork S}_{*}(M) \hookrightarrow C_{*}(M)$ is a chain deformation retraction and their homology groups are isomorphic.
\end{proof}

We now define our main objects of study.
\begin{definition}
If $W \in S$ then $\tau_{W} \in C^{*}_{\pitchfork S}(M)$ is the cochain defined by $\tau_W (\sigma) = \# P,$  where $P$ is
the  zero-manifold defined as the following pullback
$$
\begin{CD}
P @>>> W\\
@VVV  @VfVV\\
\Delta^n @>\sigma>> M.
\end{CD}
$$
The cardinality $\# P$ is to be taken as mod-two unless $M$ and $W$ are oriented, in which case $\Delta^n$ is given the  orientation with say the differences between vertices labelled by $1, \cdots, n$ and vertex $0$ (or the barycenter) serving as basis vectors, and $P$ is then oriented accordingly.
\end{definition}

We call these Thom classes, because in the standard development they are (the pushforwards of)  Thom classes of the normal bundles.  But we will use 
them to understand both pushforward and Thom classes, rather than the other way around.

\begin{theorem}
If $\partial V = W$ then $\delta \tau_{V} = \tau_{W} \in C^{*}_{\pitchfork \{V, W\}}$.  In particular, if $V$ has no boundary than $\tau_{V}$ is a cocycle.
\end{theorem}

\begin{proof}[Sketch, for now]
This comes down to the classification of one-manifolds.  Consider $\sigma : \Delta^{n+1} \to M$.  
Look at pull back of $W$ through $\sigma$ to get a one-manifold with boundary.  Some boundary points are on $\partial \Delta^{n+1}$ - 
counting those gives 
 $\delta \tau_V (\sigma)  = \tau_V (\partial \sigma)$.  Other boundary points are in the interior, which come from the preimage of 
$\partial V = W$, and thus correspond to $\tau_{W} (\sigma)$.
\end{proof}

By abuse, we denote by $[\tau_{W}]$ the corresponding singular cohomology class under the isomorphism established in Theorem~\ref{T:qi}.

\begin{corollary}
If $\partial V = W \sqcup W'$ then $[\tau_W] = [\tau_{W'}]$.
\end{corollary}

This theorem and its
corollary recovers a classical view of cohomology, as represented by submanifolds with the relation defined by submanifolds with boundary.  But in fact
not all cocycles are represented in this way, which was a famous question of Steenrod addressed by Renee Thom.  Our theory allows for linear combinations
of submanifolds, and even manifolds with corners below, in which case all cohomology is representable.  The submanifold point of view led Thom
to initiate cobordism theory, which gives generalized cohomology theories.

Induced maps are a basic ingredient of cohomology theory.  In the case of Thom classes, they are geometrically defined on the cochain level.  The
proof of the following is immediate.

\begin{proposition}\label{pullback}
If $f : M \to N$ is transverse to $W \subset N$ then $f^{\#}(\tau_{W}) = \tau_{f^{-1} W}$.
\end{proposition}



\section{Intersection and cup product}

In this section we prove the following.

\begin{theorem}
If $V$ and $W$ intersect transversally, and the intersection is given the orientation... then $[\tau_{V}] \smile [\tau_W] = [\tau_{V \cap W}]$.
\end{theorem}

Recall that the cup product is essentially induced by the diagonal map $X \overset{\Delta}{\to} X \times X$.  Under the standard
formulaic approach, the cohomology of a product $X \times Y$ is developed through the external cup product, namely $C^*(X) \otimes C^*(Y)$
maps to $C^*(X \times Y)$ by sending $\alpha  \otimes \beta$ to $p^* \alpha \cup q^*\beta$, where $p$ is the projection from 
$X \times Y$ to $X$ and $q$ is the projection to $Y$.

\begin{proposition}\label{externalcup}
The external cup product sends $[\tau_V] \otimes [\tau_W]$ to $[\tau_{V \times W}]$
\end{proposition}

\begin{proof}[Proof of the Fundamental Theorem, based on Proposition~\ref{externalcup}]
If $V \pitchfork W$ then $\Delta \pitchfork V \times W \subset X \times X$.  Apply Proposition~\ref{pullback}.
\end{proof}

\begin{proof}[Proof of Proposition~\ref{externalcup}]
Recall K\"unneth theorem that  $C_*(X) \otimes C_*(Y) \to C_*(X \times Y)$ given by (subdividing) products of simplices in $X$ and $Y$ is an
isomorphism on homology.  Then show that $p^* \tau_V \cup q^* \tau_W$ and $\tau_{V \times W}$ agree on this subcomplex, which is has an immediate
geometric proof.
\end{proof}

\section{Examples}

We make some calculations 
It is more standard to establish fundamental classes in homology, rather than Thom classes.  But we see that Thom classes are just as natural if not moreso.
The following is immediate from the definitions.

\begin{proposition}
The value of a Thom class on  a fundamental class given by triangulating a submanifold is given by intersection.
\end{proposition}


\subsection{Projective spaces}

\subsection{Grassmannians}

\subsection{Compliments of submanifolds which are boundaries}

\subsection{Configuration spaces}

\section{Applications: wrong-way maps, duality, suspensions and Thom isomorphism}

\section{Looking forward: characteristic classes, loop spaces, and cobordism.}

\end{document}

 
© 2018 GitHub, Inc.
Terms
Privacy
Security
Status
Help
Contact GitHub
API
Training
Shop
Blog
About
Press h to open a hovercard with more details.
